\chapter{Обзор} \label{ch:ch1}

Краткая вводная
В 1847 году Дж.~Г.~Стокс \cite{Stokes} впервые показал, что при распространении нелинейной волны длиной  , циклической частотой   и волновым числом   по поверхности жидкости жидкие частицы перемещаются в направлении распространения волны со средней скоростью пропорциональной квадрату амплитуды   и экспоненциально убывая с глубиной  :
.
Это явление впоследствии получило название «дрейф Стокса». Несмотря на свою давнюю историю этот феномен активно исследуется до сих пор и находит применение в самых неожиданных областях, например в исследовании гидромагнитного динамо \cite{herreman2011stokes}. Недавнее исследование \cite{eames1999connection} показало, что дрейф Дарвина \cite{darwin1953note} и дрейф Стокса имеют одну и ту же природу. Также очень ярким примером может служить недавнее исследование \cite{feng2010ocean}, показавшее, что лангусты вида Panulirus Cygnus сохраняют свою популяцию около берегов Австралии только благодаря существованию дрейфа Стокса. Учет дрейфа Стокса позволил создать корректную модель годичных перемещений Panulirus Cygnus. В работе \cite{monismith2007hydrodynamics} проведен обзор работ по исследованию течений над коралловыми рифами, одним из выводов является то, что дрейф Стокса может составлять существенную величину течениях над рифами. В исследовании \cite{rohrs2014wave} отмечено, что дрейф Стокса способствует существованию выделенного направления дрейфа икры и личинок морских животных , в частности арктической трески.

Исобе и соавторы \cite{isobe2014selective} исследовали перенос частичек мезопластика и микропластика во Внутреннем Японском море, связанный с совместным действием дрейфа Стокса и terminal velocity. Исследование выглядит еще более актуальным на фоне работы \cite{karami2018microplastic}, в которой показано наличие микропластиковых и мезопластиковых частиц в консервированной рыбе.

Кумар и Феддерсен попытались учесть дрейф Стокса в моделях, описывающих движение жидкости в пришельфовой зоне в случае нестратифицированной \cite{kumar2017effect} и стратифицированной \cite{kumar2017effectb} жидкости. и показано, что влияние дрейфа Стокса значительно сказывается на распределение температуры воды. (каким образом учтен дрейф Стокса --- непонятно, ни слова о математической формулировке модели)

Курцик, Чень и Озгокмен наблюдали за волнами в океане, индуцированными ураганным ветром (ураганом Исаак) в Мексиканском заливе \cite{curcic2016hurricane} и обнаружили, что дрейф Стокса составляет около 20 процентов от всего потока Лагранжа и может менять направление потока до 20 градусов. Сделан вывод о важности учета дрейфа Стокса в моделях переноса вещества в верхних слоях океана, особенно в присутствии сильных ветров.

Дж. Г. Стокс в своей работе анализировал распространение гравитационных волн по поверхности идеальной жидкости. Впервые попытку учесть влияние вязкости сделал Харрисон в 1908 году \cite{harrison1909influence}. Он рассматривал движение во втором порядке малости по амплитуде волнового движения в описании Эйлера. Харрисон также в своей работе учитывал только гравитационные силы.

Несложным образом тот же результат можно получить и для капиллярно-гравитационных волн \cite{lamb1932hydrodynamics} с отличием только в дисперсионном уравнении. Для гравитационных волн взаимосвязь между циклической частотой, волновым числом и другими параметрами задачи выглядит  , а для капиллярно-гравитационных волн, распространяющихся по границе жидкости с коэффициентом поверхностного натяжения   и плотностью  :  .

После работы Харрисона, видимо в связи с мировыми войнами, наблюдалось некоторое затишье в исследовании этого явления до появления фундаментальной и наиболее цитируемой работы Лонге-Хиггинса \cite{longuet1953mass}. Лонге-Хиггинс предложил учитывать влияние вязкости на движение жидкости следующим образом: жидкость разбивается на тонкий приповерхностный слой толщиной  , в котором учитывается вихревое движение, возникающее из-за влияния вязкости и внутренний слой жидкости, полагающийся идеальным. 
Задача решается в переменных Эйлера и для описания скорости дрейфа используется переход от скорости в переменных Эйлера   к описанию скорости в переменных Лагранжа  :
(*)
Основное внимание в статье уделено описанию скорости движения в приграничном слое около дна. Было получено выражение для скорости в придонном слое и на нижней границе внутреннего слоя для бегущей и стоячей волны. Скорость внутренней части жидкости около дна описывается выражением:
.
Для граничного слоя около свободной поверхности представлена формула для расчета градиента, но отсутствует формула для расчета скорости. Сравнение с экспериментальными данными в работе производится с классической формулой, полученной Стоксом (или ее обобщением см. Ламб).

Тем самым Лонге-Хиггинс открыл новую страницу в истории изучения явления, связанную с построением теории пограничного слоя вблизи дна и вблизи свободной поверхности жидкости.

Наряду с работами Стокса \cite{Stokes} и Лонге-Хиггинса \cite{longuet1953mass, longuet1960mass} одной из ключевых работ по развитию теории дрейфового движения, инициированного волновым движением по поверхности жидкости является работа Пирсона [Pierson, 1962]. Пирсон впервые применил переменные Лагранжа для записи и решения уравнений движения жидкости. Впоследствии исследователи долго будут спорить, какой способ записи уравнений более удобен и выгоден для решения задачи расчета дрейфа.

Кратко результат полученный Пирсоном представлен ниже: в этой работе был применен метод малых возмущений для решения уравнения Навье-Стокса, записанного в переменных Лагранжа. Жидкость полагалась вязкой и несжимаемой. Сами уравнения были сформулированы Ламбом \cite{lamb1932hydrodynamics}. Уравнение непрерывности в случае, когда метки жидких частиц совпадают с начальными координатами примет вид:

Здесь  ,   и   – переменные Лагранжа (у Пирсона обозначены  ,   и  ). Уравнение движения запишется:
(1)
Здесь  ,  , а  ,   и   – это компоненты вектора скорости жидкой частицы. Домножая уравнения (1)на  ,   и   соответственно, затем на  ,   и   и на  ,   и  получим:
(2)
Запись оператора Лапласа   в форме Лагранжа для составляющей скорости   выглядит следующим образом:
(3)
Аналогичным образом выглядит действие оператора для составляющих скорости   и  .
Малые колебания расписываются около положения  ,  ,   и   в виде:
(4)
Подставляя (3) и (4) в (2) и выделяя линейные слагаемые по   получим формулировку в первом порядке малости:
(5)
Здесь  .
Решения задачи первого порядка малости (5) подставляется в задачу второго порядка малости. Уравнения второго порядка малости у Пирсона записаны двумя разными способами, которые можно свести друг к другу:

(6)

Или:

(7)

К обеим системам уравнений (6) и (7) необходимо добавить уравнение непрерывности:
(8)



Теория пограничного слоя и возникающие противоречия.
После работ Лонге-Хиггинса различные исследователи ставили эксперименты с целью подтвердить теорию Лонге-Хиггинса \cite{longuet1953mass}. В некоторых экспериментах было обнаружено придонное течение, которое не объясняется теорией Лонге-Хиггинса. Был создан ряд работ направленных на объяснение этого явления \cite{Dore and others}. 

В работе \cite{longuet1960mass} Лонге-Хиггинс большее внимание уделяет пограничному слою около свободной поверхности вязкой жидкости. Было получено выражение для скорости внутри приповерхностного слоя:
.
Снова большое внимание уделяется градиенту скорости около свободной поверхности. В частности Лонге-Хиггинс обратил внимание, на несоответствие полученного им градиента скорости и значения, полученного Харрисоном в \cite{harrison1909influence}. Лонге-Хиггинс объяснил это несоответствие неверной трактовкой Харрисоном граничного условия на давление на свободной поверхности. В представленном сравнении с экспериментом наблюдаются очень большие погрешности, что не позволяет с полной уверенностью утверждать что формула, предложенная Лонге-Хиггинсом корректно описывает ситуацию.

В дальнейшем исследователи посвятили очень много времени и сил построению теории пограничного слоя вблизи свободной поверхности. Вокруг этой проблемы в 70-е годы 20 века развернулась очень серьезная дискуссия

Чанг опубликовала работу, в которой сделана попытка обобщить результат Лонге-Хиггинса на произвольный поток волн в океане \cite{chang1969mass}. Для этого используется спектральное представление волн. В отличие от Лонге-Хиггинса Чанг решает задачу в описании Лагранжа, используя разложение Пирсона. Полученный в работе результат согласуется в предельном переходе с результатом Стокса для скорости дрейфа идеальной жидкости, инициированного распространением периодической волны. Также выполняется согласование с результатом , полученным Лонге-Хиггинсом для скорости придонного дрейфа и градиента скорости около свободной поверхности. Однако представленные в работе формулы и математический аппарат запутаны и трудны для конечного использования. В частности, необходимо иметь априорное знание о спектральном представлении волн на поверхности для того, чтобы иметь возможность рассчитать скорость дрейфа. Также используемый в работе математический аппарат довольно громоздок и проследить за математическими выкладками оказывается непростой задачей. А экспериментальные исследования в работе сравниваются с теоретическими значениями, полученными для идеальной жидкости, а экспериментальные погрешности таковы, что не позволяют говорить о хорошем соотношении теории и эксперимента. Эта работа была раскритикована Хуангом \cite{huang1970discussion}. Хуанг в своей заметке указал на математическую ошибку в размышлениях Чанг. Также в другой своей работе \cite{huang1970mass} проводил схожий с Чанг анализ, но использовал описание Эйлера и пришел к выводу, что градиент скорости около свободной поверхности совпадает с представленным Харрисоном, и отмечает, что подход, используемый Лонге-Хиггинсом дает бесконечный дрейф в глубокой жидкости. Позднее Унлуата и Мэй \cite{unluata1970mass} и Dore \cite{dore1973mass} укажут на математические ошибки в работе Хуанга \cite{huang1970mass} и покажут согласование своих результатов с результатом Лонге-Хиггинса \cite{longuet1953mass}. Также Чанг \cite{chang1970reply} покажет справедливость математических выводов  в предыдущей работе \cite{chang1969mass}.

Унлуата и Мэй \cite{unluata1970mass} в своей работе рассматривают задачу в постановке, предложенной Лонге-Хиггинсом \cite{longuet1953mass, longuet1960mass} и пытаются построить теорию пограничного слоя. Решая задачу в переменных Лагранжа и используя разложение Пирсона по малому параметру. Показано, что при помощи стандартной теории возмущений задачу не решить для бесконечно глубокого океана и необходимо учитывать большее значение толщины пограничного слоя. В этом случае задача решается и градиент скорости около свободной поверхности должен совпадать с приведенным Лонге-Хиггинсом. 

Мадсен \cite{Madsen} используя те же методы, что и Унлуата и Мэй \cite{unluata1970mass} рассматривает учетв лияния ветра и силы Кориолиса на дрейфовые течения в вязком глубоком океане. 

Развитие это исследование получило уже в 1994 году \cite{xu1994wave}. Ксу и Боуэн решали задачу в переменных Эйлера и учитывали силу Кориолиса и влияние ветра на дрейфовое движение в водоемах с различным профилем дна. 

Очень большой вклад в развитие теории пограничного слоя внес Дор. В работе \cite{dore1970mass} он решает задачу по расчету скорости массопереноса в двухслойной системе вязких жидкостей в приближении волн малой амплитуды. Задача решается в переменных Эйлера во втором порядке малости по амплитуде волнового движения по границе раздела жидкостей. Жидкости как и у Лонге-Хиггинса делятся на вязкие пограничные слои и внутреннюю идеальную часть получены выражения для скорости дрейфа во внутренних областях и в пограничных слоях жидкостей. 

Работа \cite{dore1973mass} является расширением \cite{dore1970mass}. Здесь автор применяет ортогональную криволинейную систему координат, движущуюся с волной (что позволяет записывать граничные условия сразу на границе раздела). Рассматриваются бегущие и стоячие волны, когда верхняя граница ограничена твердой поверхностью, в случае, когда верхняя граница – свободная поверхность дополнительно рассматривается взаимодействие между волнами. Обе жидкости раскладываются на пограничные слои и внутренние области, соответственно накладываются ограничения на длину волны ( ) и толщину жидкого слоя ( ). Пренебрегается пространственными вязкими эффектами (пространственным затуханием и изменением фазовой скорости из-за влияния вязкости). Для пересчета скорости Эйлера в скорость Лагранжа как и у Лонге-Хиггинса используется формула (*).

В работе \cite{dore1974mass} рассмотрено обобщение решения, полученного Лонге-Хиггинсом \cite{longuet1953mass} на трехмерный случай. Получены формулы для скорости дрейфа во внутренней области при распространении двух волн под углом   друг к другу. Задача также решается в криволинейных ортогональных координатах, движущихся с волной.

В работе \cite{dore1977mass} основной упор делается на расчет дрейфа для бегущих волн большой амплитуды (для стоячих волн на свободной поверхности и стоячих волн на границе раздела теория представлена в работах \cite{dore1976double,dore1976double2}). В основном анализ проводился для волн на поверхности бесконечно глубокой жидкости. 

В работе \cite{dore1978some} обобщается результат Лонге-Хиггинса \cite{longuet1953mass} на случай границы, вода-воздух и рассматривается изменение профиля скорости дрейфа, связанного с влиянием воздуха на свободную поверхность. Важным результатом является обнаружение того, что для границы вода-воздух, влияние ветра можно не учитывать только для волн с длиной порядка метров и более.

В \cite{dore1978double} используются те же методы, что и в \cite{dore1977mass}, но для волн бегущих по границе раздела двух полубесконечных жидкостей (вода-воздух). 

Лиу и Дэвис \cite{liu1977viscous} рассматривались как стоячие так и бегущие волны в слабо вязкой жидкости, для затухающих со временем волн. Задача решалась методом малых возмущений и методом многих масштабов в переменных Эйлера. В отличие от работ Лонге-Хиггинса, полученные решения справедливы при любых (а не только очень больших) числах Рейнольдса.
Показано, что учет убывания волн незначительно влияет на распределение скорости в пограничных слоях, но оказывает заметное влияние на скорость во внутренней области. Результаты лучше согласуются с экспериментом Рассела и Осорио \cite{russell1957experimental}.

Крэйк в соей работе \cite{craik1982drift} рассматривает затухающие со временем волны, решая ту же задачу, что и Лиу и Дэвис в работе \cite{liu1977viscous}, чье решение содержало неточности. Также им посчитана скорость устойчивого дрейфового течения для пространственно затухающих волн и показано сходство с решением Лонге-Хиггинса для незатухающих волн. 
Задача решается в приближении малой вязкости в криволинейных переменных Эйлера. Рассматривается случай, когда по поверхности распределена нерастяжимая пленка (учет производится только в поверхностных напряжениях). показано, что в случае, когда на поверхности распределена нерастяжимая пленка граничное условие, взятое из Лонге-Хиггинса \cite{longuet1953mass} не применимо ни для каких длин волн. И предлагается другое граничное условие, учитывающее временное затухание движения. Для чистой свободной поверхности, по которой распространяются затухающие со временем волны получено решение, не содержащее сингулярностей (как это было у Лиу и Дэвиса) из-за применения корректного граничного условия.
Также Крэйк отмечает отсутствие адекватного эксперимента, позволяющего проследить за профилем дрейфа Стокса: «A truly definitive experiment on drift profiles is still lacking, so long after Stokes’s pioneering paper».

Искандарани и Лиу в начале 90-х годов \cite{iskandarani1991mass} предложили численную модель, позволяющую рассчитать массоперенос вязкой жидкости в присутствии на дне резервуара объекта, частично отражающего волну. Модель работает в переменных Эйлера и справедлива для произвольных значений толщины пограничного слоя. Позднее ими была предложена трехмерная модель расчета массопереноса, в которой в отличии от двумерной модели обнаружена сильная зависимость скорости массопереноса от толщины пограничного слоя жидкости \cite{iskandarani1991massb}.



Учет различных факторов.
Новый взгляд на проблему возникает с появлением работы Эндрюса и Макинтайра \cite{andrews1978exact}. Они предложили новый математический аппарат, позволяющий определять скорость дрейфового движения. 

Развитие этого математического аппарата позволило совершить скачок в исследовании дрейфа Стокса. Исследователи расширили применение теории дрейфа Стокса и рассматривали такие проблемы как вертикальный дрейф Стокса в сжимаемой жидкости (атмосфере) \cite{coy1986stokes}.

Активно этот аппарат использовал в своих работах Я. Э. Вебер для учета влияния различных факторов на скорость дрейфового движения. 

В работе \cite{weber1983attenuated} Вебер рассматривает дрейф, вызванный поверхностными волнами в вязком вращающемся бесконечно глубоком океане. Задача решалась в переменных Лагранжа во втором порядке малости по амплитуде волны, использовалось разложение из работы Пирсона \cite{pierson1962perturbation}. Усредняя уравнения движения по пространственным координатам на длине волны, записываются уравнения для средней дрейфовой скорости жидких частиц. В отличие от предшественников рассматриваются затухающие волны (нет притока энергии в волновое движение). Показано также, что градиент скорости дрейфа во внутренней области около пограничного слоя на свободной поверхности совпадает по значению с полученным Лонге-Хиггинсом. Нет попыток сравнить полученное решение с экспериментами Лонге-Хиггинса \cite{longuet1960mass} или Рассела и Осорио \cite{russell1957experimental} под предлогом того, что статья предназначена для того, чтобы показать, что ни классическая теория Стокса (даже модифицированная вязкостью) ни существующие на тот момент эксперименты не могут быть использованы для описания дрейфа, возникающего в океане (показывается важность одновременного учета вращения и вязкости). Получены выражения для комплексной скорости,которые в частном случае сводятся к выражениям, полученным Лонге-Хиггинсом \cite{longuet1969nonlinear}, также получены выражения для скоростей в приближении малого времени наблюдения.

В работе \cite{weber1983steady} исследуется незатухающие поверхностные волны в бесконечно глубоком вязком вращающемся океане. В отличие от предшественников здесь допускается изменение поверхностного натяжения за счет воздействия ветра. За счет этого компенсируются потери энергии за счет вязкой диссипации. 

В работе \cite{weber1987wave} Вебер изучал дрейф, индуцированный волновым движением в условиях слабовязкого вращающегося океана, поверхность которого покрыта тонким слоем шугового льда. Задача решается без учета поверхностного натяжения и в приближении  . Рассчитываются средние дрейфовые течения в океане под слоем льда. Также было определено среднее вязкое напряжение, действующее на лед и показано, что оно может быть сопоставимо по величине с напряжением, вызванным ветрами умеренной силы. 
Суммарный поток под слоем льда (положительный) отличается от суммарного потока в свободном океане (равен нулю). Это означает наличие струйного течения под слоем льда и может стать причиной возникновения восходящих потоков.

В работе \cite{weber1989effect} рассматривается влияние поверхностной пленки на скорость дрейфа, инициированного капиллярно-гравитационным волновым движением.
Рассматривалась вязкая бесконечно глубокая несжимаемая жидкость, по поверхности которой распространяются бегущие, убывающие со временем (в отличии от Крэйка, который рассматривал неубывающие волны) волны. Свободная поверхность жидкости покрыта тонкой нерастворимой и нерастяжимой в тангенциальном направлении пленкой поверхностно-активного вещества (ПАВ). Задача формулируется и решается в переменных Лагранжа во втором порядке малости по амплитуде волнового движения. 
Для затухающих со временем волн получено выражение для скорости дрейфа. Проводится сравнение со скоростью в отсутствии пленки и в пренебрежении влияния воздуха. Разница в скорости дрейфа очень сильно зависит от длины волны чем длиннее волны – тем заметнее разница, для капиллярных волн разницы практически нет. Влияние воздуха должно быть похожим на влияние пленки ПАВ, но должно слабее сказываться.

В работе \cite{weber1990effect} рассматривали влияние воздуха на скорость дрейфа, инициированного волновым движением по поверхности жидкости. Задача решалась в описании Лагранжа. Обе жидкости считались полубесконечными, вязкими и однородными. Считалось, что по границе раздела распространяется бегущая капиллярно-гравитационная волна, затухающая пространственно или со временем, а вся система вращается вокруг вертикальной оси с постоянной угловой скоростью. Верхняя жидкость имеет плотность много меньшую, чем нижняя. Получено выражение для скорости дрейфа, инициированного затухающими со временем волнами на границе раздела и для пространственно затухающих волн.

В \cite{weber1993transient} рассматривается распространение бегущей волны по границе раздела вода-воздух. Обе жидкости предполагаются полубесконечными, однородными, несжимаемыми, а вся система, вращающейся вокруг вертикальной оси с постоянной угловой скоростью. Решается двумерная задача в переменных Лагранжа. Для решения используется разложение по малому параметру, предложенное Пирсоном \cite{pierson1962perturbation}. Упор сделан на рассмотрении гравитационных волн, генерируемых ветром на границе раздела сред. Получено выражение для скорости среднего дрейфа. Оно для удобства понимания физического смысла разделено на 4 составляющих части: дрейф, обусловленный внешним средним касательным напряжением сдвига, классический дрейф Стокса, модифицированный вязким затуханием волны, поправка, вносимая вязким пограничным слоем с завихренностью, и квази-Эйлеровый индуцированный волновым движением поток (вне пограничного слоя он равен суммарному потоку, индуцированному волновым движением минус невязкий дрейф Стокса). 
Численно получены результаты для некоторых случаев: для волн затухающих со временем, для пространственно затухающих волн и для нарастающих волн (видимо имеется в виду неустойчивость Кельвина-Гельмгольца, но в явном виде это нигде не сказано).

Работа \cite{weber1995effect} направлена на исследование влияния пленки, распределенной по поверхности жидкости на скорость дрейфа, инициированного волновым движением.
В работе рассматривалась однородная несжимаемая вязкая бесконечно глубокая жидкость, по поверхности которой распределена тонкая нерастворимая пленка. Считалось, что по поверхности жидкости распространяется двумерная бегущая волна. Модуль поверхностной упругости пленки   определяется следующим образом \cite{lucassen1969eh}:

Здесь   – коэффициент поверхностного натяжения, а   – площадь, приходящаяся на молекулу материала пленки. 
Поверхностное натяжение   раскладывается в ряд по малому параметру   (отклонение концентрации от своего равновесного значения):

В статье рассматриваются свободно распространяющиеся и затухающие волны (не рассматривается генерация волн). Пренебрегается влиянием верхней среды, т.е. над поверхностью жидкости предполагается вакуум и, как следствие, пренебрегается нормальными и тангенциальными напряжениями на поверхности. Задача решается в переменных Лагранжа. Для решения используется метод малых возмущений, предложенный Пирсоном \cite{pierson1962perturbation}. Также по амплитудному малому параметру расписывается концентрация пленки поверхностно-активного вещества (ПАВ). Решение линейной задачи взято из \cite{weber1993transient} 
Во втором порядке малости по амплитуде волны было получено выражение для скорости дрейфа. Также как и в случае нерастяжимой пленки, эффект пленки заключается в увеличении скорости дрейфа на поверхности в небольшом интервале времени. При более длительном наблюдении более сильное затухание приводит к уменьшению дрейфовой скорости. Таким образом, итоговое расстояние, пройденное жидкой частицей в присутствии пленки будет больше, чем на чистой поверхности только при малых значениях эластичности. 
Сравнение с экспериментом не производилось («Unfortunately, we can find no experimental evidence for this kind of behavior. It appears that no controlled drift experiment up to now has been conducted with a film-covered area that is large compared to the square of the wavelength, say. Drift observations in wave tanks of small oil lenses, plastic floats, or oleyl alcohol slicks, as reported in Refs. 24 and 25, do not meet our requirements of a continuous film cover.»)
Основная цель работы – учесть эластичность пленки ПАВ. Показано, что для нерастворимой пленки учет элластичности заметно влияет на скорость дрейфа. Это связано с взаимодействием двух волновых режимов: основного и волн Марангони. Наибольшее затухание происходит, когда частота волнового движения близка к частоте волн Марангони.






Дрейф Стокса. Общие вопросы. ПАВ.
Практически с самого начала интерес исследователей был сконцентрирован на учете различных факторов влияющих на дрейф Стокса или на расчете дрейфа для разнообразных типов волн. Очень большое внимание оказывалось присутствию пленки ПАВ на свободной поверхности жидкости и рассмотрение влияния волнового движения и пленки друг на друга. Лонге-Хиггинс положил начало изучению дрейфа Стокса применительно к разнообразным системам и типам волн, например для двойных волн Кельвина \cite{longuet1969transport}.
В работе \cite{longuet1969wave} Лонге-Хиггинс рассматривает новый механизм генерации волн в океане и предлагает концепцию виртуальных волновых напряжений на свободной поверхности. 
Впоследствии эта концепция будет активно использоваться Вебером. В работе [Weber, 2001] она применена к различным приложениям, таким как пространственно затухающие волны или волны в присутствии пленки ПАВ на свободной поверхности жидкости. И рассчитаны средние скорости течения жидкости.

Экспериментаторы также не могли не обратить внимание на исследование влияния ПАВ на волновое движение. Так в работе \cite{huhnerfuss1981damping} исследуется затухание волн под действием ПАВ. Однако ошибки эксперимента таковы, что можно судить лишь о качественном согласовании моделей с результатами опыта.

Лоу в своей работе \cite{law1999wave} исследует дрейф поверхностной нерастяжимой пленки аналитически и экспериментально. Аналитический анализ проводился в криволинейных координатах. Функция тока вычислена до третьего порядка малости по крутизне волны. Пограничный слой полагается очень малым в сравнении с амплитудой волны. Показано, что аналитические выражения в \cite{law1999wave} и [Филипс, 1980]  недооценивают дрейф.

Кристенсен и Вебер развивают исследование Лоу и в работе \cite{christensen2005drift} сравнивают экспериментальные данные, представленные в \cite{law1999wave} и теоретические значения скорости дрейфа нерастяжимой пленки, представленные в работе \cite{weber1987wave}.  Однако сталкиваются с принципиальными трудностями в сравнении результатов: «Because most experimental data, like those reported by Law, are given as time averages, it has not been possible to compare the predicted time development of the drift with observations. Data on the temporal variation of the wave-induced surface drift is desirable, and we strongly recommend that this is taken into account in future laboratory studies.»
Получено, что решение, зависящее от времени (Вебер) предсказывает большие скорости, чем устойчивые (независящие от времени) решения, представленные в Филипсе [Филипс, 1980] и Лоу \cite{law1999wave}. И тем самым лучше согласуется с экспериментальными данными \cite{law1999wave}.


Канг и Ли \cite{kang1996prediction} рассмотрели аналитическую модель расчета скорости дрейфа в вязком пограничном слое, предложенную в Филипсе [Филипс, 1980]. Получили полуэмпирическую формулу для расчета скорости дрейфа. И получили следующие результаты: модель Стокса довольно хорошо прогнозирует скорость дрейфа твердых частиц, движущихся по поверхностным волнам. Если поверхностный пограничный слой считается нерастяжимым, то формула, предложенная в Филипсе справедлива только для слоев, длина которых меньше половины длины волны. Для пленок других длин предложена полуэмпирическая формула для расчета скорости дрейфа.

Пиедра-Куэва  рассматривает влияние вязкости на дрейфовые свойства пространственно затухающих волн в системе двух вязких жидкостей \cite{piedra1995drift,piedra1996drift}. Производится сравнение с экспериментом \cite{sakakiyama1989mass}, но результат сравнения не позволяет говорить о хорошем соответствии теории и эксперимента. 

Кристенсен \cite{christensen2005drift,christensen2005wave} исследовал вязкую жидкость, покурытую упругой эластичной пленкой ПАВ. Задача решалась в переменных Лагранжа, для записи использовалось разложение Пирсона \cite{pierson1962perturbation}. Исследовались пространственно-затухающие капиллярно-гравитационные волны. Показано, что существующая линейная теория хорошо согласуется с экспериментами, однако отсутствуют эксперименты для проверки нелинейной теории.

Кристенсен и Террил \cite{christensen2009drift} теоретически исследовали дрейф и деформацию нефтяной пленки, связанные с дрейфом Стокса. Показано, что важно в моделях смещения нефтяных разливов учитывать не только влияние ветра, но и дрейфовые движения, инициированные волновым возмущением свободной поверхности жидкости. Рассматривалось трехмерное волновое движение и его влияние на форму и круглого нефтяного пятна и скорость дрейфа. Не учитывалось влияние ветра, как это делается в других исследованиях.

В работе \cite{weber2006eulerian} обсуждается сравнение Эйлерова и Лагранжевого подходов к описанию движения для расчета среднего потока, индуцированного волновым движением. рассматривается вязкая полубесконечная жидкость со свободной поверхностью, вращающаяся вокруг вертикальной оси. Влияние ветра учтено в напряжении на поверхности. Одним из результатов является то, что интегрирование уравнения для импульса в переменных Эйлера от постоянной глубины, на которой исчезает движение до свободной поверхности дает те же уравнения для переноса жидких частиц, что и полученные от анализа в переменных Лагранжа во втором порядке малости по крутизне волны.




Траектории движения индивидуальных жидких частиц.
Параллельным курсом шло изучение траекторий движения индивидуальных жидких частиц. Здесь история снова начинается с работ Лонге-Хиггинса. Понимание того, что направление дрейфа связано с направлением вращательнго движения жидких частиц по своим орбитам пришло уже в 1969 году \cite{longuet1969nonlinear}. Однако пионерские работы по расчету траекторий движения появились значительно позже \cite{longuet1979trajectories,longuet1986eulerian}. Лонге-Хиггинс обратил внимание на то, что период волнового движения (Эйлеров период) и период вращения жидкой частицы вокруг среднего положения (Лагранжев период) не совпадают и привел теорию для расчета этих периодов. Однако долгое время в этой работе обращали внимание на другой ее аспект: а именно разницу в фиксируемых Эйлеровом и Лагранжевом уровне жидкости. 

К 2000-м годам появляется осознание того, что полноты картины необходимо описать не только скорость движения жидкости и построить теорию пограничного слоя, но и понять как движутся индивидуальные жидкие частицы, участвующие в волновом движении. Для расчета траекторий и скорости движения индивидуальных жидких частиц задача формулируется и решается в основном в переменных Эйлера. Так в 2001 году А. Константин рассмотрел волны Герстнера на поверхности полубесконечной жидкости и получил выражения для траекторий движения индивидуальных жидких частиц \cite{constantin2001deep}. А в 2006 году он рассмотрел движение гравитационных волн Стокса по поверхности идеальной несжимаемой жидкости конечной толщины, расположенной на твердой подложке. И математически доказал, что траектории движения жидких частиц незамкнуты \cite{constantin2006trajectories}. Этими исследованиями он положил начало новому этапу исследований. Для расчета траекторий движения индивидуальных жидких частиц исследователи, как правило, решают задачу в системе отсчета, движущейся в направлении распространения волны с фазовой скоростью волны. 

В работах Генри приводится доказательство незамкнутости траекторий движения жидких частиц при распространении по свободной поверхности идеальной бесконечно глубокой жидкости волн Стокса \cite{henry2006trajectories} и для капиллярных и капиллярно-гравитационных линейных волн \cite{henry2007particle,henry2007particle2}.

Константин и Виллари \cite{constantin2008particle,constantin2008particleDeepWater} рассматривали траектории жидких частиц в линейных волнах на поверхности воды и получили схожие результаты с представленными в \cite{constantin2006trajectories}. 

Эхрнстрём рассмотрел влияние завихренности на траектории движения индивидуальных жидких частиц в бесконечно глубокой \cite{ehrnstrom2007deep} и в жидкости конечной глубины \cite{ehrnstrom2008streamlines}. Получены условия на существование дрейфа, направленного в сторону распространения волны и незамкнутости траекторий движения жидких частиц. В работе \cite{ehrnstrom2009recent} показано, что в устойчивых волнах траектории жидких частиц незамкнуты и присутствует положительно направленный дрейф в случае нулевой или отрицательной завихренности. Также внимание уделено солитонам и стоячим волнам.


Хсу \cite{hsu2013particle} предложил решение третьего порядка малости по амплитуде волны в переменных Лагранжа для идеальной жидкости, по поверхности которой распространяется гравитационная волна. Однако этот метод неприменим в более общем случае двух контактирующих идеальных жидкостей.


С развитием компьютерной техники стали активно развиваться численные методы. Так в 2007 году Чанг, Лиу и Су представили метод численного расчета траекторий движения жидких частиц в жидкости с постоянной глубиной \cite{chang2007particle} для гравитационных волн конечной амплитуды. В 2009 году Чанг, Чень и Лиу \cite{chang2009particle} разработали численный метод расчета траекторий движения жидких частиц в бесконечно глубокой жидкости и обратили внимание на несогласование периодов волнового движения и циклического движения индивидуальной жидкой частицы. Годом позже Чень и соавторы предложили эксперимент \cite{chen2010lagrangian}, в котором демонстрируется это различие. А Окамото и Шоджи \cite{okamoto2012trajectories} используя численные методы производят расчет траекторий движения жидких частиц для волн Краппера (волны в отсутствии гравитации) и впервые для капиллярно-гравитационных волн, распространяющихся по поверхности жидкости. Метод справедлив только для идеальной жидкости. Представлено доказательство теоремы, сформулированной Константином \cite{constantin2006trajectories} о незамкнутости траекторий движения индивидуальных жидких частиц для случая бесконечно глубокой жидкости, используя тот же подход, что и в \cite{constantin2006trajectories} (переход в систему отсчета, движущуюся с фазовой скоростью волны). Также представлено математическое доказательство того, что дрейф везде направлен в сторону распространения волны и максимален на свободной поверхности.




В недавнем исследовании Шнирельман \cite{shnirelman2012analyticity} строго доказал, что для идеальной несжимаемой жидкости существует аналитическое описание траекторий движения индивидуальных жидких частиц. 

Исследование траекторий движения индивидуальных жидких частиц также находит разнообразное применение. Например в \cite{santamaria2013stokes} рассматривается перемещение взвешенных мелких частиц в бесконечно глубокой вязкой жидкости, вызванное совместным действием оседания и дрейфа Стокса. А Вебер исследовал траектории движения жидких частиц во внутренних волнах имеющих тип волн Герстнера \cite{weber2018interfacial}.

В работе \cite{abd2013new} сделана попытка решить задачу о расчете траекторий движения жидких частиц с использованием алгебры Ли.


Однако исследователями строгое аналитическое описание траекторий движения индивидуальных жидких частиц даже в самых простейших случаях используется крайне редко. И совсем не уделено внимание вопросу согласования траекторий движения индивидуальных жидких частиц и отклонения свободной поверхности от равновесного положения.



Эксперименты
Харрис \cite{harris1966wave} сделал предположение, что при распространении волнового движения по границе жидкой и газообразной среды в последней тоже должен наблюдаться перенос вещества, аналогичный дрейфу Стокса и проверил его экспериментально. В прямоугольном канале с водой продуцировалось волновое движение. В камеру запускался дым и фотографировался в разные моменты времени. Было обнаружено смещение дыма в сторону распространения волны.

Рассел и Озорио \cite{russell1957experimental} исследовали скорость дрейфа в прямоугольном канале и следили за ее профилем при помощи измерения скорости мелких жидких частиц, взвешенных в воде. Для некоторых длин волн решение Лонге-Хиггинса лучше описывает результат эксперимента, но для глубокой воды наилучшим оказывается описание Стокса.

В работе \cite{tsuchiya1980mass} исследовались перманентные волны в прямоугольном канале, профиль волны, амплитуда и скорость распространения волны измерялись при помощи двух емкостных детекторов, а скорость движения жидких частиц измерялась при помощи фотографии на длинной выдержке. Результаты эксперимента лучше согласуются с теорией Стокса. Лучшее согласование с теорией Лонге-Хиггинса ожидается на более длительных временах наблюдения, когда вязкость сыграет более значительную роль.

В 2000 году исследователи провели эксперимент по влиянию неоднородности дна, оказываемого на дрейф, индуцированный волновым движением \cite{ridler2000effect}. Измерения проводились в прямоугольном канале при помощи доплеровского радара. Показано, что в зависимости от неоднородности дна придонные течения могут быть как больше, так и меньше тех, которые предсказываются Лонге-Хиггинсом для случая плоского дна.

В 2016 норвежскими исследователями был проведен эксперимент по определению траекторий движения индивидуальных жидких частиц, их скорости и периода вращательного движения в прямоугольном канале с жидкостью, используя метод слежения за жидкими частицами (particle tracking velocimetry) \cite{grue2017experimental}. Было обнаружено, что существует два уровня, на которых скорость дрейфа обращается в нуль, период кругового движения жидкой частицы сравнивается с периодом волнового движения, а траектории становятся замкнутыми.

В работе \cite{umeyama2011measurements} экспериментально исследуется скорость и траектории движения индивидуальных жидких частиц в двухслойной жидкости. Результаты сравниваются с численным расчетом и показывают удовлетворительное сходство.

Используя доплеровский радар в работе \cite{marin2004eddy} исследовался дрейф и влияние формы дна на дрейф в пограничном придонном слое.



Вывод
Проблема расчета дрейфа, инициированного волновым движением поверхности жидкости не оставила равнодушными множество исследователей из различных стран по всему миру. 
Несмотря на многообразие теорий, на практике чаще всего используются простые формулы, полученные еще в середине прошлого века, в силу сложности математического аппарата и трудности повторения выкладок. 
Хотя вопрос о волновом движении по границе раздела двух жидких сред возникал и рассматривался, остался незамеченным один очень важный вопрос: на настоящий момент отсутствует простая модель, позволяющая рассчитать среднее дрейфовое движение и траектории движения индивидуальных жидких частиц в модели двухслойной идеальной жидкости, в том числе в случае, когда один слой движется относительно другого с постоянной скоростью. Возможно свою роль сыграло то, что очень многие исследователи используют переменные Лагранжа для описания движения, однако в этом случае не удается корректно записать граничные условия на границе раздела двух идеальных жидкостей.
Несмотря на то, что кажется очевидным, что средний дрейф складывается из малых смещений жидких частиц при петлеобразных движениях и существует целый ряд исследований по расчету траекторий движения индивидуальных жидких частиц или мелкодисперсного мусора в жидкости, исследователями оказался незамечен и не поднят вопрос о согласовании движения жидкой частицы находящейся на поверхности и профиля волнового движения. 
