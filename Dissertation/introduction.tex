\chapter*{Введение}                         % Заголовок
\addcontentsline{toc}{chapter}{Введение}    % Добавляем его в оглавление

\newcommand{\actuality}{}
\newcommand{\progress}{}
\newcommand{\aim}{{\textbf\aimTXT}}
\newcommand{\tasks}{\textbf{\tasksTXT}}
\newcommand{\novelty}{\textbf{\noveltyTXT}}
\newcommand{\influence}{\textbf{\influenceTXT}}
\newcommand{\methods}{\textbf{\methodsTXT}}
\newcommand{\defpositions}{\textbf{\defpositionsTXT}}
\newcommand{\reliability}{\textbf{\reliabilityTXT}}
\newcommand{\probation}{\textbf{\probationTXT}}
\newcommand{\contribution}{\textbf{\contributionTXT}}
\newcommand{\publications}{\textbf{\publicationsTXT}}


{\actuality} В~самых разнообразных и порой неожиданных академических, биофизических, технических геофизических приложениях исследователи сталкиваются с~проблемой массопереноса, инициированного волновым движением вдоль поверхности жидкости. Волновой массоперенос имеет самое непосредственное отношение к~проблемам расчета переноса загрязнения по поверхности океана; транспортировки примеси в~многослойных структурах, формирующихся как в~атмосфере, так и в~океане; к~вопросам моделирования закономерностей миграции некоторых видов флоры и фауны. Особый интерес исследователей связан с~переносом поверхностно--активных веществ (ПАВ) вдоль поверхности жидкости и влиянием плёнки ПАВ на~переносные свойства волн. В~связи с~этим не менее значимыми оказываются вопросы разработки методов мониторинга и управления местоположением областей загрязнения поверхности открытых водоемов и его влияния на~динамику волнового движения. Исследование влияния растворимых и нерастворимых плёнок ПАВ на~волновое движение на~поверхности жидкости тесно связано с~одной стороны с~перспективой формирования более полного понимания физической сути явления, а с~другой~--- для разработки новых методов управления условиями и характером протекания различного рода неустойчивостей, реализующихся на~поверхности жидкости.
 Несмотря на~давнюю историю вопроса и большое количество исследований в~этой области, в~практических приложениях дрейф, связанный с~волновым возмущением поверхности жидкости, учитывается при помощи модели, предложенной Дж.Г.~Стоксом еще в~середине XIX века. До сих пор не предложено простой аналитической, удобной для применения на~практике процедуры расчета скорости дрейфа Стокса в~многослойных системах жидкостей. В~начале XXI столетия активизировались экспериментальные и теоретические исследования в~области построения траекторий индивидуальных частиц жидкости, причем теоретические работы направлены в~основном на~использование численных методов расчета. 
\ifsynopsis

\else
Этот абзац появляется только в~диссертации.
Через проверку условия \verb!\!\verb!ifsynopsis!, задаваемого в~основном файле
документа (\verb!dissertation.tex! для диссертации), можно сделать новую
команду, обеспечивающую появление цитаты в~диссертации, но~не~в~автореферате.
\fi

% {\progress}
% Этот раздел должен быть отдельным структурным элементом по
% ГОСТ, но он, как правило, включается В~описание актуальности
% темы. Нужен он отдельным структурынм элемементом или нет ---
% смотрите другие диссертации вашего совета, скорее всего не нужен.

{\aim} данной работы является аналитическое асимптотическое исследование основных закономерностей волнового массопереноса в~идеальных и вязких жидкостях, а также изучение влияния тангенциального разрыва скоростей, поверхностного электрического заряда и плёнки поверхностно--активного вещества на~скорость дрейфа и характер движения индивидуальных жидких частиц, участвующих в~периодическом и переносном движениях.

Для~достижения поставленной цели были решены следующие {\tasks}:
\begin{enumerate}
  \item Разработана теоретическая аналитическая асимптотическая методика расчета траекторий движения индивидуальных частиц жидкости, участвующих в~волновом движении.
  \item Теоретическое аналитическое исследование влияния тангенциального разрыва скоростей на~границе раздела двух идеальных жидкостей на~закономерности движения индивидуальных жидких частиц и на~скорость массопереноса.
  \item Теоретическое аналитическое исследование влияния поверхностного электрического заряда на~закономерности движения индивидуальных жидких частиц и на~скорость массопереноса.
  \item Теоретическое аналитическое исследование влияния амплитудной модуляции волнового движения, оказываемое на~закономерности движения индивидуальных жидких частиц и на~скорость массопереноса.
  \item Теоретическое аналитическое исследование влияния плёнки поверхностно--активного вещества на~закономерности движения индивидуальных жидких частиц и на~скорость массопереноса в~вязких жидкостях.
  \item Теоретическое аналитическое исследование влияния волнового движения поверхности вязкой жидкости на~характер распределения поверхностно--активного вещества вдоль верхней границы жидкости.
\end{enumerate}


{\novelty}
\begin{enumerate}
  \item Впервые была разработана аналитическая асимптотическая методика расчета траекторий движения индивидуальных жидких частиц, участвующих в~волновом движении жидкости. Методика позволяет используя простые аналитические выражения построить траектории движения индивидуальных частиц жидкости с~учетом  факторов, оказывающих влияние на~круговую частоту волнового движения (например тангенциальный разрыв скоростей или поверхностный электрический заряд) и модулирующих амплитуду.
  \item Впервые было обнаружено новое явление неустойчивости Кельвина--Гельмгольца, заключающееся в~возникновении дрейфовых течений в~контактирующих жидкостях, направленных таким образом, чтобы уменьшить тангенциальный разрыв скоростей, инициировавший неустойчивость.
  \item Впервые аналитически во втором приближении по амплитуде волны описан характер движения индивидуальных жидких частиц в~присутствии поверхностно--активного вещества на~поверхности жидкости.
  \item Впервые аналитически было получено описание перераспределения  концентрации поверхностно--активного вещества, связанного с~волновым возмущением поверхности жидкости. Были получены зависимости положения максимума концентрации ПАВ от упругости плёнки.
  \item Впервые аналитически во втором приближении по амплитуде волны получены скорости дрейфового движения жидкости в~присутствии поверхностно--активного вещества на~её поверхности. Выделены составляющие дрейфового течения, связанные с~действием вдоль направления распространения волны горизонтальных вязких напряжений. 
\end{enumerate}

{\influence} работы состоит в~том, что полученные результаты представляют собой теоретическую основу для дальнейшего развития теоретических представлений о дрейфовых движениях, инициированных волновым движением вдоль поверхности жидкости, о траекториях индивидуальных жидких частиц, формирующих дрейфовое и циклическое движения, о характере перераспределения поверхностно--активных веществ вдоль поверхности жидкости. В~результате работы было обнаружено новое, неизвестное до этого явление неустойчивости Кельвина--Гельмгольца, которое представляет интерес для приложений,  имеющих дело с~системой двух жидкостей, испытывающих тангенциальный разрыв скоростей вдоль границы раздела. Полученные результаты по определению характера перераспределения плёнки ПАВ вдоль поверхности жидкости могут быть применены в~задачах мониторинга и прогнозирования распространения нефтяных разливов (и других поверхностно--активных веществ) в~мировом океане. В~работе развита новая методика расчета траекторий движения индивидуальных частиц жидкости, которая позволяет при помощи несложных выражений получить представления о переносе вещества волнами и может быть использована в~самых разнообразных метеорологических, биофизических, геофизических, технических, технологических и научных приложениях. 

{\methods} заключается в~использовании стандартных аналитических асимптотических методов математической физики, метода разложения по малому параметру. При решении были использованы классические модели гидродинамики и электрогидродинамики.

{\defpositions}
\begin{enumerate}
%  \item Разработана аналитическая асимптотическая методика перехода от описания поля скоростей в~переменных Эйлера к~описанию в~переменных Лагранжа. Методика позволяет совершать аналитический асимптотический переход с~учетом горизонтальных сдвиговых движений жидкостей.
%  \item Обнаружено свойство неустойчивости Кельвина--Гельмгольца, заключающееся в~формировании дрейфовых течений в~контактирующих жидкостях, направленных таким образом, чтобы скомпенсировать тангенциальный разрыв скоростей, инициировавший неустойчивость.
%  \item Аналитически получены результаты расчета влияния поверхностного электрического заряда на~скорость дрейфа и траектории движения индивидуальных частиц жидкости, связанные с~распространением по поверхности жидкости капиллярно--гравитационной волны. Поверхностный электрический заряд уменьшает скорость дрейфовых движений, за счет уменьшения круговой частоты волнового движения (увеличения эйлерова периода) и увеличивает лагранжев период волнового движения (уменьшает частоту обращения индивидуальной частицы жидкости вокруг среднего положения). 
%  \item Аналитически определено влияние амплитудной модуляции капиллярно--гравитационного волнового возмущения поверхности жидкости, на~скорость инициируемого дрейфа и траектории движения индивидуальных частиц жидкости. При распространении волнового пакета вдоль поверхности жидкости средние дрейфовые течения оказываются примерно в~двое меньшими по сравнению с~течениями, связанными с~распространением простейшей синусоидальной волны. 
%  \item Аналитически установлено влияние плёнки поверхностно--активного вещества на~траектории движения индивидуальных частиц жидкости. С увеличением упругости плёнки ПАВ уменьшается внутренняя площадь траектории, заметаемой индивидуальной жидкой частицей за период. При достижении упругостью плёнки ПАВ характерного значения, траектории вырождаются в~отрезки прямых, наклоненных к~горизонту под углом около $ \pi/4 $. Дальнейший рост упругости приводит к~возникновению круговых движений с~изменением направления обхода траектории.
%  \item Разработано аналитическое представление о перераспределении поверхностно--активного вещества, связанного с~распространением капиллярно--гравитационной волны по поверхности вязкой жидкости. Предложенные формулы позволяют проследить за положением максимума концентрации плёнки ПАВ, покрывающей вязкую жидкость в~зависимости от упругости и связать это распределение с~декрементами затухания капиллярно--гравитационных волн, распространяющихся вдоль поверхности жидкости.
%  \item Аналитически определены скорости дрейфового течения, инициируемого волновым движением вдоль поверхности вязкой жидкости, покрытой плёнкой поверхностно--активного вещества. Выделены составляющие дрейфового движения, затухание которых с~глубиной носит экспоненциальный характер и являющиеся преемственными классическому дрейфу Стокса и составляющие, связанные с~наличием упругих напряжений между слоями вязкой жидкости. Определено влияние ПАВ на~эти компоненты скорости дрейфа.
% 
  	  \item Результаты разработки новой аналитической асимптотической методики перехода от описания поля скоростей в~переменных Эйлера к~описанию в~переменных Лагранжа. Методика позволяет совершать аналитический асимптотический переход с~учетом горизонтального сдвига жидкостей.
  	\item Новое свойство неустойчивости Кельвина--Гельмгольца, заключающееся в~формировании дрейфовых течений в~контактирующих жидкостях, направленных таким образом, чтобы скомпенсировать тангенциальный разрыв скоростей, инициировавший неустойчивость.
  	\item Результаты аналитического расчета влияния поверхностного электрического заряда на~скорость дрейфа и траектории движения индивидуальных частиц жидкости, связанные с~распространением по поверхности жидкости капиллярно--гравитационной волны.
  	\item Результаты аналитического расчета влияния амплитудной модуляции капиллярно--гравитационного волнового возмущения поверхности жидкости, на~скорость инициируемого дрейфа и траектории движения индивидуальных частиц жидкости.
  	\item Результаты анализа влияния плёнки поверхностно--активного вещества на~траектории движения индивидуальных частиц жидкости.
  	\item Результаты разработки нового аналитического представления о перераспределении поверхностно--активного вещества, связанного с~распространением капиллярно--гравитационной волны по поверхности вязкой жидкости.
  	\item Результаты аналитического расчета скорости дрейфового течения, инициируемого волновым движением вдоль поверхности вязкой жидкости, покрытой плёнкой поверхностно--активного вещества.
   
\end{enumerate}

{\reliability} полученных результатов обеспечивается предельными переходами к~известным аналитическим выражениям, использованием апробированных методов и аналитических подходов к~решению задач. Результаты находятся в~соответствии с~экспериментальными результатами, и результатами численных моделирований полученными другими авторами.


{\probation}
Основные результаты работы докладывались~на:
VIII, X, XI и XII международной конференции <<Волновая электрогидродинамика проводящей жидкости. Долгоживущие плазменные образования и малоизученные формы естественных электрических разрядов в~атмосфере>> (Ярославль, 2009, 2013, 2015, 2019); IX международной конференции <<Современные проблемы электрофизики и электродинамики жидкостей>> (Петергоф, 2009); XI и XII международной конференции <<Современные проблемы электрофизики и электрогидродинамики>> (Петергоф, 2015, 2019); II, III, IV, V, VI, VII международной научно~-- практической конференции <<Путь в~науку. Физика>> (Ярославль, 2014, 2015, 2016, 2017, 2018, 2019); всероссийской молодёжной  научной конференции <<Путь в~науку. Математика>> (Ярославль, 2019); 67 региональной научно~-- технической конференции студентов, магистрантов и аспирантов высших учебных заведений с~международным участием (Ярославль, 2014); всероссийской школе~-- семинаре <<Волны~--- 2016>>, <<Волны~--- 2017>>, <<Волны~--- 2018>>, <<Волны~--- 2019>> (Можайск, 2016, 2017, 2018, 2019); ХХI всероссийской школе~-- конференции молодых ученых <<Состав атмосферы. Атмосферное электричество. Климатические процессы>> (Борок, 2017); международной конференции «Динамические системы в~науке и технологиях» (DSST-2018) (Алушта, 2018); 9-ой международной конференции~-- школе молодых ученых <<Волны и вихри в~сложных средах>> (Москва, 2018); VI международной конференции <<Актуальные проблемы механики сплошной среды>> (Дилижан, Армения, 2019).

{\contribution} Автор выполнял постановки решаемых задач, принимал активное участие в~обсуждении и интерпретации результатов исследования, самостоятельно производил аналитические и численные вычисления, докладывал результаты работы на~многочисленных конференциях.

\ifnumequal{\value{bibliosel}}{0}
{%%% Встроенная реализация с~загрузкой файла через движок bibtex8. (При желании, внутри можно использовать обычные ссылки, наподобие `\cite{vakbib1,vakbib2}`).
    {\publications} Основные результаты по теме диссертации изложены в~XX печатных изданиях,
    X из которых изданы в~журналах, рекомендованных ВАК,
    X "--- в~тезисах докладов.
}% 
{%%% Реализация пакетом biblatex через движок biber
    \begin{refsection}[bl-authorvak,bl-authorwos,bl-authorscopus,bl-authorother,bl-authorconf]
        % Это refsection=1.
        % Процитированные здесь работы:
        %  * подсчитываются, для автоматического составления фразы "Основные результаты ..."
        %  * попадают в~авторскую библиографию, при usefootcite==0 и стиле `\insertbiblioauthor` или `\insertbiblioauthorgrouped`
        %  * нумеруются там в~зависимости от порядка команд `\printbibliography` в~этом разделе. 
        %  * при использовании `\insertbiblioauthorgrouped`, порядок команд `\printbibliography` в~нём должен быть тем же (см. biblio/biblatex.tex)
        %
        % Невидимый библиографический список для подсчёта количества публикаций:
        \printbibliography[heading=nobibheading, section=1, env=countauthorvak,    keyword=biblioauthorvak]%
        \printbibliography[heading=nobibheading, section=1, env=countauthorwos,    keyword=biblioauthorwos]%
        \printbibliography[heading=nobibheading, section=1, env=countauthorscopus, keyword=biblioauthorscopus]%
        \printbibliography[heading=nobibheading, section=1, env=countauthorconf,   keyword=biblioauthorconf]%
        \printbibliography[heading=nobibheading, section=1, env=countauthorother,  keyword=biblioauthorother]%
        \printbibliography[heading=nobibheading, section=1, env=countauthor,       keyword=biblioauthor]%
        %
        % Цитирования.
        %  * Порядок перечисления определяет порядок в~библиографии (только внутри подраздела, если `\insertbiblioauthorgrouped`).
        %  * Если не соблюдать порядок "как для \printbibliography", нумерация в `\insertbiblioauthor` будет кривой.
        %  * Если цитировать каждый источник отдельной командой --- найти некоторые ошибки будет проще.
        %
        %% authorvak
        \nocite{vakbib1}%
        \nocite{vakbib2}%
        \nocite{vakbib3}%
        \nocite{vakbib4}%
        \nocite{vakbib5}%
        \nocite{vakbib6}%
         \nocite{vakbib7}%
        %
        %% authorwos
        \nocite{WoSbib1}%
        %
        %% authorscopus
        \nocite{scbib1}%
        \nocite{scbib2}%
        %
        %% authorconf
        \nocite{conf1}%
        \nocite{conf2}%
        \nocite{conf3}%
        \nocite{conf4}%
        \nocite{conf5}%
        \nocite{conf6}%
        \nocite{conf7}%
        \nocite{conf8}%
        \nocite{conf9}%
        \nocite{conf10}%
        \nocite{conf11}%
        \nocite{conf12}%
        \nocite{conf13}%
        \nocite{conf14}%
        \nocite{conf15}%
        \nocite{conf16}%
        \nocite{conf17}%
        \nocite{conf18}%
        \nocite{conf19}%
        \nocite{conf20}%
        \nocite{conf21}%
        \nocite{conf22}%
        \nocite{conf23}%
        \nocite{conf24}%
        %
        %% authorother
        \nocite{bib1}%
        \nocite{bib2}%
        \nocite{bib3}%
        \nocite{bib4}%
        \nocite{bib5}%
        \nocite{bib6}%
        \nocite{bib7}%
        \nocite{bib8}%
        \nocite{bib9}%
        \nocite{bib10}%
        %
        %
        {\publications} Основные результаты по теме диссертации изложены в~\arabic{citeauthor}~печатных изданиях,
        \newcounter{citeauthorscwostot}% сумма citeauthorscopus и citeauthorwos
        \setcounter{citeauthorscwostot}{\value{citeauthorscopus}}%
        \addtocounter{citeauthorscwostot}{\value{citeauthorwos}}%
        \arabic{citeauthorvak} из которых изданы в~журналах, рекомендованных ВАК\sloppy%
        \ifnum \value{citeauthorscwostot}>0%
            , \arabic{citeauthorscwostot} "--- в~периодических научных журналах, индексируемых Web of Science и Scopus\sloppy%
        \fi%
        \ifnum \value{citeauthorconf}>0%
            , \arabic{citeauthorconf} "--- в~тезисах докладов.
        \else%
            .
        \fi
    \end{refsection}%
    \begin{refsection}[bl-authorvak,bl-authorwos,bl-authorscopus,bl-authorother,bl-authorconf]
        % Это refsection=2.
        % Процитированные здесь работы:
        %  * попадают в~авторскую библиографию, при usefootcite==0 и стиле `\insertbiblioauthorimportant`.
        %  * ни на~что не влияют в~противном случае
%            %% authorvak
%    \nocite{vakbib1}%
%    \nocite{vakbib2}%
%    \nocite{vakbib3}%
%    \nocite{vakbib4}%
%    \nocite{vakbib5}%
%    \nocite{vakbib6}%
%    %
%    %% authorwos
%    \nocite{WoSbib1}%
%    %
%    %% authorscopus
%    \nocite{scbib1}%
%    \nocite{scbib2}%
%    %
%    %% authorconf
%   \nocite{conf1}%
%   \nocite{conf2}%
%   \nocite{conf3}%
%   \nocite{conf4}%
%   \nocite{conf5}%
%   \nocite{conf6}%
%   \nocite{conf7}%
%   \nocite{conf8}%
%   \nocite{conf9}%
%   \nocite{conf10}%
%   \nocite{conf11}%
%   \nocite{conf12}%
%   \nocite{conf13}%
%   \nocite{conf14}%
%   \nocite{conf15}%
%    %
%    %% authorother
%    \nocite{bib1}%
%    \nocite{bib2}%
%    \nocite{bib3}%
%    \nocite{bib4}%
%    \nocite{bib5}%
%    \nocite{bib6}%
%    \nocite{bib7}%
%    \nocite{bib8}%
    \end{refsection}%
	%
	% Всё, что вне этих двух refsection, это refsection=0,
	%  * для диссертации - это нормальные ссылки, попадающие в~обычную библиографию
	%  * для автореферата:
	%     * при usefootcite==0, ссылка корректно сработает только для источника из `external.bib`. Для своих работ --- напечатает "[0]" (и даже Warning не вылезет).
	%     * при usefootcite==1, ссылка сработает нормально. В~авторской библиографии будут только процитированные в~refsection=0 работы.
}

% %% authorvak
\nocite{vakbib1}%
%\nocite{vakbib2}%
%\nocite{vakbib3}%
%\nocite{vakbib4}%
%\nocite{vakbib5}%
%\nocite{vakbib6}%
%%
%%% authorwos
%\nocite{WoSbib1}%
%%
%%% authorscopus
%\nocite{scbib1}%
%\nocite{scbib2}%
%%
%%% authorconf
%\nocite{conf1}%
%\nocite{conf2}%
%\nocite{conf3}%
%\nocite{conf4}%
%\nocite{conf5}%
%\nocite{conf6}%
%\nocite{conf7}%
%\nocite{conf8}%
%\nocite{conf9}%
%\nocite{conf10}%
%\nocite{conf11}%
%\nocite{conf12}%
%\nocite{conf13}%
%\nocite{conf14}%
%\nocite{conf15}%
%%
%%% authorother
%\nocite{bib1}%
%\nocite{bib2}%
%\nocite{bib3}%
%\nocite{bib4}%
%\nocite{bib5}%
%\nocite{bib6}%
%\nocite{bib7}%
%\nocite{bib8}%
%%

 % Характеристика работы по структуре во введении и в автореферате не отличается (ГОСТ Р 7.0.11, пункты 5.3.1 и 9.2.1), потому её загружаем из одного и того же внешнего файла, предварительно задав форму выделения некоторым параметрам

\textbf{Объем и структура работы.} Диссертация состоит из~введения, четырёх глав и~заключения.
%% на случай ошибок оставляю исходный кусок на месте, закомментированным
%Полный объём диссертации составляет  \ref*{TotPages}~страницу
%с~\totalfigures{}~рисунками и~\totaltables{}~таблицами. Список литературы
%содержит \total{citenum}~наименований.
%
Полный объём диссертации составляет
\formbytotal{TotPages}{страниц}{у}{ы}{}, включая
\formbytotal{totalcount@figure}{рисун}{ок}{ка}{ков}
 %и \formbytotal{totalcount@table}{таблиц}{у}{ы}{}
 .   Список литературы содержит
\formbytotal{citenum}{наименован}{ие}{ия}{ий}.
