%% Согласно ГОСТ Р 7.0.11-2011:
%% 5.3.3 В заключении диссертации излагают итоги выполненного исследования, рекомендации, перспективы дальнейшей разработки темы.
%% 9.2.3 В заключении автореферата диссертации излагают итоги данного исследования, рекомендации и перспективы дальнейшей разработки темы.
\begin{enumerate}
	\item Разработана аналитическая асимптотическая методика перехода от описания поля скоростей в переменных Эйлера к описанию в переменных Лагранжа. Методика позволяет совершать аналитический асимптотический переход с учетом горизонтальных сдвиговых движений жидкостей.
	\item Обнаружено свойство неустойчивости Кельвина--Гельмгольца, заключающееся в формировании дрейфовых течений в контактирующих жидкостях, направленных таким образом, чтобы скомпенсировать тангенциальный разрыв скоростей, инициировавший неустойчивость.
	\item Аналитически получены результаты расчета влияния поверхностного электрического заряда на скорость дрейфа и траектории движения индивидуальных частиц жидкости, связанные с распространением по поверхности жидкости капиллярно--гравитационной волны. Поверхностный электрический заряд уменьшает скорость дрейфовых движений, за счет уменьшения круговой частоты волнового движения (увеличения эйлерова периода) и увеличивает лагранжев период волнового движения (уменьшает частоту обращения индивидуальной частицы жидкости вокруг среднего положения). 
	\item Аналитически определено влияние амплитудной модуляции капиллярно--гравитационного волнового возмущения поверхности жидкости, на скорость инициируемого дрейфа и траектории движения индивидуальных частиц жидкости. При распространении волнового пакета вдоль поверхности жидкости средние дрейфовые течения оказываются примерно в двое меньшими по сравнению с течениями, связанными с распространением простейшей синусоидальной волны. 
	\item Аналитически установлено влияние плёнки поверхностно--активного вещества на траектории движения индивидуальных частиц жидкости. С увеличением упругости плёнки ПАВ уменьшается внутренняя площадь траектории, заметаемой индивидуальной жидкой частицей за период. При достижении упругостью плёнки ПАВ характерного значения, траектории вырождаются в отрезки прямых, наклоненных к горизонту под углом около $ \pi/4 $. Дальнейший рост упругости приводит к возникновению круговых движений с изменением направления обхода траектории.
	\item Разработано аналитическое представление о перераспределении поверхностно--активного вещества, связанного с распространением капиллярно--гравитационной волны по поверхности вязкой жидкости. Предложенные формулы позволяют проследить за положением максимума концентрации плёнки ПАВ, покрывающей вязкую жидкость в зависимости от упругости и связать это распределение с декрементами затухания капиллярно--гравитационных волн, распространяющихся вдоль поверхности жидкости.
	\item Аналитически определены скорости дрейфового течения, инициируемого волновым движением вдоль поверхности вязкой жидкости, покрытой плёнкой поверхностно--активного вещества. Выделены составляющие дрейфового движения, затухание которых с глубиной носит экспоненциальный характер и являющиеся преемственными классическому дрейфу Стокса и составляющие, связанные с наличием упругих напряжений между слоями вязкой жидкости. Определено влияние ПАВ на эти компоненты скорости дрейфа.
	
	
%  \item Разработана аналитическая асимптотическая методика расчета траекторий движения индивидуальных жидких частиц, участвующих в циклическом и дрейфовом движениях, связанных с распространением волны или волнового пакета Стокса вдоль поверхности жидкости.
%  \item Обнаружено новое свойство неустойчивости Кельвина--Гельмгольца, заключающееся в формировании дрейфовых течений в контактирующих жидкостях, направленных таким образом, чтобы уменьшить тангенциальный разрыв скоростей, инициировавший развитие неустойчивости.
%  \item Обнаружено значение упругости пленки ПАВ, при котором затухание волнового движения наиболее эффективно. Получено положение максимума концентрации вещества ПАВ в зависимости от упругости пленки.   
%  \item Показано влияние упругости ПАВ на характер движения индивидуальных жидких частиц. 
%  \item Получены аналитические асимптотические выражения для определения скорости дрейфа Стокса вязкой жидкости, покрытой пленкой поверхностно--активного вещества.
\end{enumerate}
